\documentclass[a4paper,12pt]{article}

\usepackage{times}                       % 使用 Times New Roman 字体
\usepackage{CJK,CJKnumb,CJKulem}         % 中文支持宏包
\usepackage{color}                       % 支持彩色

%——————————–其他宏包——————————–
%\usepackage{amsmath,amsthm,amsfonts,amssymb,bm} % 数学宏包
%\usepackage{graphicx,psfrag}                    % 图形宏包
%\usepackage{makeidx}                            % 建立索引宏包
%\usepackage{listings}                           % 源代码宏包

%———————————正文———————————–
\begin{document} % 开始正文
\begin{CJK*}{GBK}{song}                           % 开始中文环境

\author{pcghost}                                 % 作者
\title{latex模板}                                % 题目
\maketitle                                       % 生成标题

TEX是由图灵奖得主Knuth编写的计算机程序,用于文章和数学公式的排版。

1977年Knuth开始编写TEX排版系统引擎的时候,\\ % 换行
是为了探索当时正开始进入出版工业的数字印刷设备的潜力。他特别希望能因此扭转那种排版质量下降的趋势,
使自己写的{\CJKfamily{hei}书和文章}免受其害。

\clearpage % 换页,\newpage也可以,推荐\clearpage
我们现在使用的TEX系统是在1982年发布的,1989年又略作改进,增进了
对8字节字符和多语言的支持。TEX以具有优异的稳定性,可以在各种不同
类型的计算机上运行,以及几乎没有错误而著称。
\end{CJK*}     % 结束中文环境
\end{document} % 结束正文